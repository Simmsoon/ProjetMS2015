% definit le type de document et ses options
\documentclass[a4paper,11pt]{article}

% des paquetages indispensables, qui ajoutent des fonctionnalites
\usepackage[latin1]{inputenc}
\usepackage[T1]{fontenc}
\usepackage{amsmath,amssymb}
\usepackage{fullpage}
\usepackage{graphicx}
\usepackage{url}
\usepackage{xspace}
\usepackage[francais]{babel}

\newtheorem{partie}{Partie}

% des commandes utiles pour ecrire des maths : rajoutez les votres!
\newcommand{\dx}{\,dx}
\newcommand{\ito}{,\dotsc,}
\newcommand{\R}{\mathbb{R}}
\newcommand{\N}{\mathbb{N}}
\newcommand{\Poly}[1]{\mathcal{P}_{#1}}
\newcommand{\abs}[1]{\left\lvert#1\right\rvert}
\newcommand{\norm}[1]{\left\lVert#1\right\rVert}
\newcommand{\pars}[1]{\left(#1\right)}
\newcommand{\bigpars}[1]{\bigl(#1\bigr)}
\newcommand{\set}[1]{\left\{#1\right\}}

% titre, auteur et date
\title{TP Principes et M\'{e}thodes Statistiques}
\author{Gabriel Sarrazin, Nejmeddine Douma, Simon Rabourg}
\date{Avril 2015}

% le debut du contenu
%===============
\begin{document}
%===============

% pour afficher titre, auteur et date
\maketitle

\section{Analyse des d\'{e}fauts de cuves}


\section{V\'{e}rifications exp\'{e}rimentales \`{a} base de simulations}


\begin{partie}
Il est possible de simuler n \'{e}chantillons de la loi Pa(a,b) car nous connaissons sa fonction de r\'{e}partiton. 
\\
Pa(a,b) est une fonction continue, elle peut donc s'apparenter \`{a} une loi uniforme.  Dans un premier temps, simuler n \'{e}chantillons de cette loi va donc consister  \`{a} tirer, au hasard, n valeurs al\'{e}atoires sur l'intervalle [0,1]. Connaissant la fonction de r\'{e}partition de la loi Pa(a,b),nous allons ensuite calculer l'image inverse F-1(Ui) pour obtenir un \'{e}chantillon de loi Pa(a,b) et nous ferons selons cela pour les n valeurs obtenues sur [0,1]
\\
Nous pouvons repr\'{e}senter cette m\'{e}thode sous forme d'un graphique : en mettant en ordonn\'{e}e les n valeurs de la loi Ui et en abscisse la projection pour chacune de ses valeurs de son image inverse (F-1(Ui) ).
\end{partie}
% fin du document
%=============
\end{document}
%=============
